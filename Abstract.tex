The determination of the specific angular momentum radial profile, $j(r)$, in the early stages of star formation is crucial to constrain the theories of star and cirumstellar disk formation.
The specific angular momentum is directly related to the maximum disk size possible and 
it can be used to constrain the mechanism for angular momentum removal. 
We present the determination of the specific angular momentum radial profile towards 2 Class 0 objects and a First Hydrostatic Core candidate. 
All three objects are close to edge-on and therefore issues of projection effect are diminished. 
We find that the $j(r)$ is consistent across all three sources and it is well fitted with a single power-law relation between 7,000 and 1,000 au: 
$j(r) = 10^{-3.5\pm 0.7}\left( \frac{r}{1,000 \textrm{au}}\right)^{1.86\pm0.18} \rm{km\, s^{-1} pc}$~.
This power-law relation is right in between the relation expected for solid body rotation ($\propto r^2$) and pure turbulence ($\propto r^{1.5}$). 
This strongly suggest that even at the scales of 1,000 au, the effect of the dense core initial level of turbulence is still strong. 
Moreover, we do not find a region of conserved specific angular momentum, although it could be found at 
smaller scales. 
Finally, these results suggest that the initial conditions for numerical simulations of disk formation 
should be modified in order to either: 
a) use a larger simulation box to include some level of driven turbulence, or 
b) incorporate the observed specific angular momentum profile to setup the initial conditions of the dense core kinematics.