\section{Discussion}

\subsection{Interpretation of the power-law}
The results are clear: the specific angular momentum radial profile is consistent with a single power-law. 
However, the power-law exponent is mid-way between the expected exponents 
for solid body rotation ($j\propto r^{2}$) and for turbulence ($j\propto r^{1.5}$). 

For solid body rotation, the velocity at distance $r$ is $v(r)=\Omega r$, 
where $\Omega$ is the angular speed, 
and therefore the specific angular momentum is $j_{SolidBody}=\Omega r^2$. 
In the case of a pure turbulent field that follows Larson's relation ($\delta v(r) \propto r^{0.5}$),  
then the dense cores kinematics (even at scales of $1,000$\,au) might still 
influenced by the turbulence of the parental molecular cloud/dense core.

This result suggests that dense cores do not present solid body rotation, as usually assumed. 
It might be an intermediate place in between turbulence dominated and solid body rotation.

\subsection{Are these results consistent with those for Dense cores?}
The previous study of dense cores by \cite{Goodman_1993} found that 
\begin{equation}
\frac{J}{M} = 10^{-0.7\pm0.2} \left(\frac{R}{\textrm 1 pc}\right)^{1.6\pm 0.2}~,
\end{equation}
where it assumed solid body rotation and uniform density. 
This relation predicts an specific angular momentum at 1,000 au of $3.95\times 10^{-5}$\,\kms pc, 
which is one order of magnitude lower than the directly measured values.

We notice that in the calculation in \cite{Goodman_1993} it is assumed a uniform density sphere 
with solid body rotation. 
We derived a more general expression for the total specific angular momentum, 
\begin{equation}
\frac{J}{M}(R) = \frac{(3-p)}{(3+q-p)} j_0 R^{q} 
{}_2F_1(1/2,−p/2,3/2,1)~.
\end{equation}
under he assumption of a density profile $\rho \propto \rho_0 r^{-p}$ and a specific angular momentum 
profile $j(r_{rot}) = j_0 r_{rot}^q$. 
This more general expression allow us to compare our results with those of \cite{Goodman_1993} and asses 
if the discrepancy arises from a .

The first thing to note is that \emph{if specific angular momentum is described as a single power-law out to the dense core radius} 
then the exponent of the total specific angular momentum is the same of the 
specific angular momentum profile. 
%
Although the power-law exponents of these relations are different, they are 
consistent within the fits' uncertainty. 
The main source of discrepancy is due to the assumption of solid body rotation.

\subsection{Implications to maximum possible disk radius}
XXX This is not very solid because of the stellar mass assumption. XXX
We estimate the maximum possible disk size using the relation
\begin{equation}
R_{disk} = \frac{j^2}{G\,M_*}~,
\end{equation}
see \cite{Yen_2015}.

For the specific angular momentum found at 1,000 au, $10^{-3.5}$\,\kms pc, and a stellar mass 
of 0.05\,M$_\odot$, which is appropriate for HH211 \citep{Lee_2009,Froebrich_2003},  
we find a disk radius of 80\,au.

This radius is already comparable to 

\subsection{The effect on numerical simulations}
A few questions while I am running some test simulations:

1) Can the power law index be reproduced by conditions other than turbulence cascade.
  e.g. changing different rotation profiles and evolve the core for a free-fall time before the protostar. 
  
2) If turbulence is required (under the condition that the most extreme rotation profile cannot reach this 1.7 power law), then what strength and power spectrum of turbulence is needed.

3) Does magnetic field play a role in shaping the 1.7 power law?
  One effect is that: magnetic braking will redistribute the angular momentum of infalling gas.
  However, it is just redistribute from equatorial plane into envelope, so can this still give the same index $q$ for $\theta$ direction?

  

  
  