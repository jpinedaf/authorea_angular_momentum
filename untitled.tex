\section{Introduction}

What is the role of angular momentum in disk formation?
Currently there are no good observational constraints on how and when a circumstellar disk, which later will form planets, gather most of its mass. 
Recently, three small surveys have attempted to study the presence and properties of these disks in the earliest stages of protostellar evolution (Class 0 stage). 
Two of them \citep{J_rgensen_2009,Enoch_2011}, found evidence for massive disks (0.02−2 \msun) and concluded that disks universally form and gather material early in the Class 0 stage. With linear resolutions ranging from 200−1,000 au, neither survey was capable of resolving disks and instead inferred their presence from compact, unresolved components in the visibilities. 
Furthermore, both surveys were biased toward fairly luminous sources and are not representative of the full range of protostellar luminosities. 
Recent observations have shown clear evidence of disks towards a few Class 0 objects \citep{Murillo_2013,Ohashi_2014}. 
However, PdBI observations of 5 Class 0 sources found no evidence for disks in most of them \citep{Maury_2010}, 
therefore suggesting that disks gather most of their mass at later stages.
The uncertainty concerning disk properties during the Class 0 stage is not limited to observational studies. 
A similar split over whether large and massive disks are possible during the Class 0 stage is found in numerical simulations. 
It all boils down to the mechanisms transporting angular momentum across the disk and their efficiency. 
Magnetic braking can efficiently remove the angular momentum, and therefore only small (few au) and low mass disks would form in the Class 0 stage \citep{Allen_2003,Hennebelle_2008,Mellon_2008,Seifried_2011}. 
However, many other potential solutions have been proposed to reduce the effect of magnetic breaking and allow disks to form, including magnetic flux loss through various mechanisms, non-ideal MHD effects, outflow-induced envelope clearing, misaligned magnetic field, and turbulence (Mellon & Li 2008; Li et al. 2011; Seifried et al. 2012). 
%%%%%
This second scenario allows the formation of big ($\sim$100 au) and massive disks during the Class 0 stage. 
In the case of massive disks, it is expected that they will promptly fragment to create a binary companion in this early stage of evolution.
It is expected that future high angular resolution observations will definitely determine the existence (or not) of disks during the Class 0 stage. 
However, this will not address the initial conditions or physics needed to reproduce observations. 
In particular, the specific angular momentum of the infalling material is directly related to the possible protostellar disk radius  and how much angular momentum must be removed to inhibit its formation.

Specific angular momentum radial profile: A determination of the specific angular momentum as a function of radius, starting from the scale of dense cores down to their inner envelope, has been challenging. 
Estimation of the specific angular momentum for dense cores have mostly been done on the largest scales, $\sim$0.1pc \citep{Goodman_1993,Caselli_2002}, while estimates at smaller scales have been done with objects with disks \citep[Class 0/I][]{1997ApJ...488..317O,Yen_2015}. 
From these heterogeneous ensemble of measurements (see Fig. 1) it is suggested \citep{Belloche_2013} that some of the specific angular momentum is lost from the largest scales, while at scales smaller than ~5,000 au the specific angular momentum is conserved until reaching the scales of disks ($\sim$100 au). 
A recent survey towards 17 Class 0/I objects using SMA \citep{Yen_2015} determined that the specific angular momentum, j, is between 10$^{-5}$-4$\times$10$^{-3}$ \kms pc at 1,000 au. This shows that the specific angular momentum present a large scatter, with values much lower than those suggested by \cite{Belloche_2013}. 
This indicates the need to carry out the proposed program to constrain this key parameter.
In addition, numerical simulations (with and without magnetic fields) need to include some initial angular momentum. While the total angular momentum is somewhat constrained, its radial distribution is not. Therefore, the determination of the angular momentum radial profile and its dependence on the environment will be crucial in understanding disks formation.

  