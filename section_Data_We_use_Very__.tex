\section{Data}

We use Very Large Array (VLA) interferometric observations of the NH$_3$(1,1) line in (23.6944955\,GHz)
compact (D) configuration. 
All sources are in the Perseus molecualr cloud (250\,pc) and the spectral resolution achieved is 0.154\,\kms. 
This line is a good tracer of the dense gas in cores \cite{Benson_1989,Goodman_1998,Tafalla_2004,Pineda_2010}.
The have already presented in other papers, although here we usually obtain smaller beam sizes than those 
used in the original papers, which allow us to probe smaller scales at the 
price of lower sensitivity for faint extended emission.

\subsection{HH211}
The observations were carried out on 2005 December 13 under project AA300. 
The original analysis of the data is presented by \cite{Tanner_2010}.
The data reduction and imaging was carried out using CASA \cite{2007ASPC..376..127M}. 
Imaging was done using a robust parameter of 0.5 and multiscale clean, which significantly reduces 
the presence of artifacts. 
Multiscale clean is used with scales of 0, 3, 9, and 27 arcsec. 
The beam size and rms levels (estimated from line free channels) are reported in Table~\ref{table:obs}.

\subsection{L1451-mm}
The observations were carried out on 2006 January 11 under project AA300. 
The original analysis of the data is presented by \cite{Pineda_2011}.
The data reduction and imaging was carried out using CASA \cite{2007ASPC..376..127M}. 
Imaging was done using a robust parameter of 0.5 and multiscale clean, which significantly reduces 
the presence of artifacts. 
Multiscale clean is used with scales of 0, 3, 9, and 27 arcsec. 
The beam size and rms levels (estimated from line free channels) are reported in Table~\ref{table:obs}.


%The phase calibrator is J0336+323, and the band pass calibrator is J0319+415.

\subsection{IRAS03282+3035 (IRAS03282)}
%For IRAS03282+3035 (hereafter IRAS03282) it is the one in \cite{Tobin_2011}. 

The observations were carried out on 2009 November 11 under project AT373. 
The original analysis of the data is presented by \cite{Tobin_2011}.
The data reduction and imaging was carried out using CASA \cite{2007ASPC..376..127M}. 
Imaging was done using a robust parameter of 0.5 and multiscale clean, which significantly reduces 
the presence of artifacts. 
Multiscale clean is used with scales of 0, 3, 9, and 27 arcsec. 
The beam size and rms levels (estimated from line free channels) are reported in Table~\ref{table:obs}.

\begin{table} 
\caption{Parameters of Interferometric Maps\label{table:obs}}
    \begin{tabular}{ c c c c c c}
        Source & RA & Dec & beamsize & rms & Outflow PA\tablefootnote{Measured East from North}\\ 
            & (hh:mm:ss) & (dd:mm:ss) & & (mJy beam$^{-1}$) & (deg) \\ 
        IRAS03282 & 3:31:20.94 & 30:45:30.3 & 3.63''$\times$2.75'' (89.56  deg) & 2.5 & 122\\ 
        HH211     & 3:43:56.52 & 32:00:52.8 & 2.63''$\times$2.55'' (-55.54 deg) & 2 & 116.6\\ 
        L1451-mm  & 3:25:10.21 & 30:23:55.3 & 3.09''$\times$2.61'' (-19.79 deg) & 3 & 10\\ 
    \end{tabular} 
\end{table}
