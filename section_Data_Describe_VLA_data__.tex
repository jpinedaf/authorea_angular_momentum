\section{Data}

We use Very Large Array (VLA) interferometric observations of the NH$_3$(1,1) line. 
This line is a good tracer of the dense gas in cores \cite{Benson_1989,Goodman_1998,Tafalla_2004,Pineda_2010}.
The have already presented in other papers, although here we usually achieve better beam sizes at the 
price of lower sensitivity for faint extended emission.

\subsection{HH211}

For HH211 it is the one in \cite{Tanner_2010}.


\subsection{L1451-mm}
The observations were carried out with VLA on 11 January 2006 under project AA300. 
The phase calibrator is J0336+323, and the band pass calibrator is J0319+415.

These observations were already presented in \cite{Pineda_2011}, but here they are 
imaged at higher angular resolution.

\subsection{IRAS03282+3035}
For IRAS03282+3035 (hereafter IRAS03282) it is the one in \cite{Tobin_2011}. 

The summary of the results is listed in Table~\ref{table:obs}

\begin{table} 
\label{table:obs}
\caption{Parameters of Interferometric Maps}
    \begin{tabular}{ c c c c c c c}
        Source & RA & Dec & channel width & beamsize & rms & Outflow PA\tablefootnote{Measured East from North}\\ 
         & (hh:mm:ss) & (dd:mm:ss) & (km s$^{-1}$) &  & (mJy beam$^{-1}$) & (deg) \\ 
        IRAS03282 & 3:31:20.94 & 30:45:30.3 & 0.154 & 3.63''$\times$2.75'' (89.56  deg) & 2.5 & 122\\ 
        HH211 & 3:43:56.52 & 32:00:52.8 & 0.154 & 2.63''$\times$2.55'' (-55.54 deg) & 2 & 116.6\\ 
        L1451-mm & 3:25:10.21 & 30:23:55.3 & 0.154 & 3.09''$\times$2.61'' (-19.79 deg) & 3 & 10\\ 
    \end{tabular} 
\end{table}
  
  
  
  
  
  
  
  
  
  
  
  
  
  
  
  
  
  
  
  
  
  
  
  
  