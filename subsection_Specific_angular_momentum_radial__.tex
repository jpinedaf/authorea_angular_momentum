\subsection{Specific angular momentum radial profile}
The specific angular momentum at a distance $r$, is defined as $j(r)=r \times \delta v$, 
where $\delta_V$ is the relative velocity of the gas with respect to the center of mass. 
%
In the case of symmetry around an axis, then the specific angular momentum can be described as, 
$j(r) = R_{rot} V_{rot}$, 
where $R_{rot}$ is the rotation radius in cylindrical coordinates (also called impact parameter) and 
$V_{rot}$ is the rotational velocity around its axis of symmetry.
Therefore, if we determine the rotational velocity at a given rotation radius we can 
derive the specific angular momentum. 
In the case of lines that are not very optically thick the rotational velocity will correspond to the 
relative centroid velocity along the line-of-sight of the line ($V_{LSR}$) with 
respect of the core center \citep{Tanner_2010}. 
A similar approach has been used to determine the velocity profile in disks \citep{Murillo_2013,2014A&A...566A..74L,Harsono_2014,Harsono_2015}.
%
In the case of a protostellar core, we use the known YSO position and outflow orientation (see Table\ref{table:obs}) 
to calculate the rotation radius.
The YSO central velocity is estimated as the velocity of the dense gas, as probed using NH$_3$, 
at the position of the YSO.
However, since we are not modeling the more complex dynamics involved in close to the disk scale, then 
we discard all determinations of the specific angular momentum at distances smaller than the beam major axis.

Since the specific angular momentum should be close to constant for a given rotational radius, 
we obtain the average the value the specific angular momentum at different rotation radii bins.
The results are shown in Fig.~\ref{Fig:summary}, with the radial profiles for all three sources, 
which appear almost indistinguishable.
The radial profiles display a distribution consistent with a power-law. 
As a comparison, we also show the profile for a core with solid body rotation down to $5,000$\,au, 
and conserved specfic angular momentum within that radius using a dash lines \citep{Belloche_2013}. 