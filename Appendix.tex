\section{Appendix}
Determination of the Total angular momentum ($J$) and specific angular momentum ($J/M$) in the case of differential rotation and/or a radial density profile.

Here, we will assume that the density is spherically symmetric and that it can be described 
by a single power-law depending only on the distance to the central protostar:
\begin{equation}%\label{eq:density}
\rho = \rho_0 r^{-p}~.
\end{equation}
Naturally, the total enclosed mass can be described as 
\begin{equation}
M(<R) = \frac{4\pi}{(3-p)} \rho_0 R^{3-p}~,
\end{equation}
for $p\ne 3$.

Also, we will assume that the specific angular momentum can be described in spherical coordinates as as
\begin{equation}
j(r,\theta) = j_0 ( r \sin \theta)^q~,
\end{equation}
where $q=2$ represents solid body rotation.
%
The total angular momentum within a radius $R$ is
\begin{eqnarray}
J(R) &=& \int_{Volume} j(r,\theta) \rho(r) d\Omega r^2 dr \\
&=& \int_0^{2\pi}\int_{0}^{\pi}\int_0^R j_0 ( r \sin \theta)^q\, \rho_0 r^{-p}\, \sin\theta r^2 dr d\theta d\phi\\
&=& 2\pi \int_{0}^{\pi} \sin^{q+1} \theta d\theta \int_0^R \rho_0 j_0 r^{2+q-p} \,dr \nonumber \\
&=& \frac{2\pi}{(3+q-p)}\rho_0 j_0 R^{(3+q-p)} \int_{0}^{\pi} \sin^{q+1} \theta d\theta~,
\end{eqnarray}
for $(p-q)\ne 3$.

Therefore, the total specific angular momentum can be written as
\begin{equation}
\frac{J}{M}(R) = \frac{(3-p)}{2(3+q-p)} j_0 R^{q} 
\int_{0}^{\pi} \sin^{q+1} \theta d\theta~.
\end{equation}


\begin{eqnarray}
\int \sin^{p+1} x dx &=& -\cos{x} _2F_1(1/2, -p/2, 3/2, \cos^2(x)) \sin^p(x) \sin^2(x)^{(-p/2)}\\
\int_{0}^{\pi} \sin^{p+1} x dx &=&  
%-( \csc(\pi/2) \cos^{(p+1)} \left(\pi/2\right) _2F_1(1/2, (p+1)/2, (p+3)/2, 0))/(p+1)
%+( \csc(-\pi/2) \cos^{(p+1)} \left(-\pi/2\right) _2F_1(1/2, (p+1)/2, (p+3)/2, 0))/(p+1)
\end{eqnarray}
  
  
  
  