\section{Results}

\subsection{Line fit}
We fit the NH$_3$ (1,1) line with all the hyperfine components using the \verb+pyspeckit+ package \cite{2011ascl.soft09001G}. 
This implements the NH$_3$ hyperfine structure fit described in \cite{Rosolowsky_2008} within \verb+python+.
This method minimizes the difference between the observed and a synthetic NH$_3$ line profile 
that is parametrized by the line centroid velocity ($V_{lsr}$), velocity dispersion ($\sigma_v$), 
kinetic temperature ($T_{kin}$), excitation temperature ($T_{ex}$), and column density ($N_{NH_3}$).
In this case we fix the kinetic temperature to 12~K, and therefore the derived column densities are not 
well constrained. 
However, this does not affect this analysis since it only relays on the kinematic properties of the gas 
which are well constrained.

For these three sources we fit the NH$_3$ (1,1) line towards all positions with a peak brightness larger than
5$\times$ rms. 
The results are shown in Figure~\ref{Fig1:HH211}, \ref{Fig1:IRAS} and \ref{Fig1:L1451}, where we 
present the integrated intensity and centroid velocity maps.
