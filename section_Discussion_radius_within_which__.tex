\section{Discussion}


radius within which the specific angular momentum is conserved. 

\subsection{Interpretation of the power-law}
This power-law is mid-way between the relations for solid body rotation ($j\propto r^{2}$) and for turbulence ($j\propto r^{1.5}$). 
For solid body rotation, the velocity at distance $r$ is $v(r)=\Omega r$, 
where $\Omega$ is the angular speed, 
and therefore the specific angular momentum is $j_{SolidBody}=\Omega r^2$. 
In the case of a pure turbulent field that follows Larson's relation ($\delta v(r) \propto r^{0.5}$), 
then the velocity at distance $r$ is $v(r)\approx \delta v(r) \propto r^{0.5}$, 
and therefore the specific angular momentum is $j_{turb}\propto r^{1.5}$ \citep{Burkert_2000}. 

Therefore, this suggest that the dynamics of the dense core, even at scales of 1,000 au, are still 
heavily influenced by the turbulence of the parental molecular cloud.

\subsection{Are these results consistent with those of Dense cores?}
aa

\subsection{The effect on numerical simulations}
The initial conditions are much different!
 

  
  