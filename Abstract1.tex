\textbf{Abstract}

The determination of the specific angular momentum radial profile, $j(r)$, in the early stages of star formation is crucial to constrain the theories of star and cirumstellar disk formation.
The specific angular momentum is directly related to the maximum disk size possible and 
it can be used to constrain the mechanism for angular momentum removal. 
We present the determination of the specific angular momentum radial profile towards 2 Class 0 objects and a First Hydrostatic Core candidate. 
All three objects are close to edge-on and therefore issues of projection effect are diminished. 
We find that the $j(r)$ is consistent across all three sources and it is well fitted with a single power-law relation: 
$j(r) = 10^{-3.5\pm 0.7}\left( \frac{r}{1,000 \textrm{au}}\right)^{1.86\pm0.18} \rm{km\, s^{-1} pc}$
  
  
  