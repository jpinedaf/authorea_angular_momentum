\section{Discussion}

\subsection{Interpretation of the power-law}
The results are clear: the specific angular momentum radial profile is consistent with a single power-law. 
However, the power-law exponent is mid-way between the expected exponents 
for solid body rotation ($j\propto r^{2}$) and for turbulence ($j\propto r^{1.5}$). 

For solid body rotation, the velocity at distance $r$ is $v(r)=\Omega r$, 
where $\Omega$ is the angular speed, 
and therefore the specific angular momentum is $j_{SolidBody}=\Omega r^2$. 
In the case of a pure turbulent field that follows Larson's relation ($\delta v(r) \propto r^{0.5}$),  
then the dense cores kinematics (even at scales of $1,000$\,au) might still 
influenced by the turbulence of the parental molecular cloud/dense core.

This result suggests that dense cores do not present solid body rotation, as usually assumed. 
It might be an intermediate place in between turbulence dominated and solid body rotation.
%
A similar exponent is also found in simulations of dense cores \citep{Dib_2010}, once cores 
have evolved and are more gravitationally bound.

\subsection{Are these results consistent with those for Dense cores?}
The previous study of dense cores by \cite{Goodman_1993} found that 
\begin{equation}
\frac{J}{M} = 10^{-0.7\pm0.2} \left(\frac{R}{\textrm 1 pc}\right)^{1.6\pm 0.2}~,
\end{equation}
where it assumed solid body rotation and uniform density. 
This relation predicts an specific angular momentum at 1,000 au of $3.95\times 10^{-5}$\,\kms pc, 
which is only a 13\% of the directly measured values reported in this work.

We notice that in the calculation in \cite{Goodman_1993} it is assumed a uniform density sphere 
with solid body rotation. 
We derived a more general expression for the total specific angular momentum, 
\begin{equation}
\label{eq:J_M}
\frac{J}{M}(R) = \frac{(3-p)\,{}_2F_1(1/2,−p/2,3/2,1)}{(3+q-p)} j_0 R^{q}~.
\end{equation}
under he assumption of a density profile $\rho \propto \rho_0 r^{-p}$ and a specific angular momentum 
profile $j(r_{rot}) = j_0 r_{rot}^q$. 
This more general expression allow us to compare our results with those of \cite{Goodman_1993} and asses 
if these results are consistent or not.

The first thing to note is that \emph{if specific angular momentum is described as a single power-law out to the dense core radius} 
then the exponent of the total specific angular momentum is the same of the 
specific angular momentum profile (eqn.~\ref{eq:J_M}). 
%
Although the power-law exponents of these relations are slightly different, 
$1.7$ compared to $1.6$, they are 
consistent within the best fit uncertainty. 
We compared the derived specific angular momentum for the entire cores, $J/M(R)$, 
with the specific angular momentum at the same scale, $j(R)$, using eqn.~\ref{eq:J_M}. 
In the case of $q=1.7$, for typical density profiles ($q=1.5$ or $2$) then 
$J/M(R)$ is $0.34\,j(R)$ and $0.25\,j(R)$, and therefore our results are within a factor of $2$
from those reported by \cite{Goodman_1993}. 

\subsection{Implications to maximum possible disk radius}
XXX This is not very solid because of the stellar mass assumption. XXX
We estimate the maximum possible disk size using the relation
\begin{equation}
R_{disk} = \frac{j^2}{G\,M_*}~,
\end{equation}
see \cite{Yen_2015}.

For the specific angular momentum found at 1,000 au, $10^{-3.5}$\,\kms pc, and a stellar mass 
of 0.05\,M$_\odot$, which is appropriate for HH211 \citep{Lee_2009,Froebrich_2003},  
we find a disk radius of 80\,au.

This radius is already comparable to 

\subsection{The effect on numerical simulations}
We have carried out a set of 3D turbulent hydrodynamical collapse simulations
with different initial rotation speed and turbulence power spectrum,
as to probe the underlying physics of the above profile
of specific angular momentum in star-froming cores.
We find that turbulence do help to reproduce a 1.7-1.8 power law for
specific angular momentum, as long as turbulence in the scale of interest
(500-20000~AU) has not yet been disspiated. We also find the a faster
rotating core tends to better match the observed profile.

We initialize an isolated prestellar cloud core with total mass
$M_c=8$\msun, and radius $R_c=4.0 \times 10^{17}$~cm
$\approx 2.67 \times 10^4$~AU. This corresponds to a initial mass density
$\rho_0=5.97 \times 10^{-20}$~g~cm$^{-3}$ and a number density for molecular
hydrogen $n({\rm H}_2) \approx 10^4$~cm$^{-3}$ (assuming a mean molecular
weight $\mu=2.36$). The free-fall time of the core is thus
$t_{\rm ff} \approx 8.6 \times 10^{12}$~s$\approx 0.27 \times 10^6$~Myr.
We assume an isothermal equation of state for the envelope gas, with
sound speed $c_s=0.2$~km~s$^{-1}$ and $T \sim 10$~K. We also impose different
solid-body rotation speed initially, as $\omega_0=2.5 \times 10^{-14}$,
$5.0 \times 10^{-14}$, and $1.0 \times 10^{-13}$; which correspoinds to
a ratio of rotational to gravitational energy
$\beta_{\rm rot}={R_c^3 \omega_0^2 \over 3 G M_c} = 0.0125$, $0.05$, and
$0.2$.

For such isolated cloud core, the profile of specific angular momentum
changes over time, yet turbulence can help to keep the profile
closer to the 1.7 power law at few 100~AU to 10$^4$~AU scale.
The stronger and longer the turbulence,
the better the reproduction of the 1.7 power law.

\noindent $\bullet$ Laminar case (Fig.~\ref{Fig:laminar}). \\
In the simplest non-turbulent case, mass distribution in the core is the main 
factor that determines the distribution of specific angular momentum. 
We can derive the following relation between index $p$ of density $\rho$ and 
index $q$ of specific angular momentum $j$, 
\begin{equation}
\label{Eq:vRprof}
v(R)=\sqrt{G M(<R) \over R}=\left[{ 4 \pi G \rho_0 \over 3-p}\right]^{1 \over 2
} R^{1-{p \over 2}}
\end{equation}
for a distribution of mass $p<3$, and $p=3$ is the case for point mass source.
Therefore, $j(R) \propto R^{2-{p \over 2}}$, i.e., $q \sim 2-{p \over2 }$. 
Note that the equation assumes that the collapse is slow and gas are stabilized 
in the current orbit.

When the initial solid body rotation still dominates the largely uniform core,
the specific angular momentum profile roughly follows a $R^2$ power law.
As collapse continues, the density profile gradually reaches
that of a Bonnor-Ebert sphere, in which $\rho \propto R^{-2}$
in the bulk of core's outer part.
Hence, the specific angular momentum in the outer part flattens to
a 1.0 power law, consistent with Eq.~\ref{Eq:vRprof} above. In the same
time, inner part of the core (inside few 100~AU) follows a power law of
$j$ steeper than 2.0, which is caused by fast collapse of high angular momentum
materials that do not have enough time to adjust their fast
rotational velocity to the current orbit. Large amount of angular
momentum piles up at few 100~AU but not yet transported into few 10~AU scale,
therefore profile of $j$ steepens. Overall, the laminar case is hard
to reproduce the 1.7 power law between
10$^2$~AU-10$^4$~AU during most of its evolution.

\noindent $\bullet$ Larson's $n=-4$ versus flat $n=0$ turbulence spectrum (Fig.~\ref{Fig:Lars1} and \ref{Fig:Mach1}).\\
Because we only initialize the turbulence at $t=0$ and no driving
is done afterwards in this isolated core, turbulence is quickly
washed out by global collapse in few 10$^3$ to 10$^4$ years. However, typical
dense cores are embedded in ambient turbulent cloud that replenishes
both the mass and turbulence in these cores.
Therefore, we investigate two the turbulence power spectra in k space
with index $n=-4$ (Larson's law) and $n=0$, to allow different lifetime
of turbulence; particularly the $n=0$ profile puts more amplitude in
small scale turbulence than the $n=-4$ profile, which can somewhat
extend the lifetime of turbulence in small scales.

In both turbulence cases, the profile of specific angular momentum
is able to sustain a power law between 1.7-1.8 for few 10$^3$ to 10$^4$ years
in the scale of interest (500-20000~AU), though in the end they
converge to the laminar case with much flatter profile ($\rightarrow$ 1.0)
after most turbulence energy is washed out by gravity.
However, we do find that the $n=0$ turbulence spectrum can sustain the
observed profile longer than the $n=-4$ case by $\sim$3 times
(3~kyr versus 10~kyr). Because the turbulence in $n=-4$ case dissipates
early in the prestellar phase, while that in the $n=0$ case is
able to last until formation of a flattened structure.
We infer that, in the presence of turbulence driving or continuous supply
of turbulence from larger cloud, the 1.7-1.8 power law can presumably be
reproduced even after formation of protostar.

The difference in initial rotation speed will be preserved at 10000-20000~AU 
throughout the entire collapse. The difference in magnitude of j 
inside 10000~AU scale becomes less obvious when enough angular momentum has 
been transported to the inner part as collapse continues. 
Overall, we find the fastest rotating model ($\beta=0.2$) can have high enough
j closer to the observed profile at 10000-20000~AU scale at all times. 
Hence, the observed sources tend to have $\beta\gtrsim0.2$, which
lies in the upper end of $\beta$ estimated by \citet{Goodman_1993}.


  

  
  